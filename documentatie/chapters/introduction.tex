\chapter*{Introducere} 
\addcontentsline{toc}{chapter}{Introducere}

Introdusă inițial ca o problemă teoretică legată de colorarea hărților geografice, colorarea grafurilor a evoluat într-un mecanism esențial de modelare și rezolvare a unei game diverse de probleme practice. Această problemă are o importanță deosebită în teoria grafurilor, ramură a matematicii discrete, și ilustrează perfect modul în care conceptele teoretice pot fi extinse și aplicate în informatică, optimizare și numeroase alte domenii științifice.
   
Colorarea grafurilor poate avea aparența unui probleme simple, asignarea de culori unor noduri ale unui graf astfel încât nodurile vecine să nu aibă aceeași culoare. Ne-am înșela însă daca am crede că este trivială, complexitatea acesteia este profundă și oferă o structură riguroasă pentru abordarea problemelor de alocare a resurselor, planificare și optimizare combinatorială.

În ultimele decenii, dezvoltarea tehnologiei și creșterea complexității sistemelor informatice au generat nevoia pentru extensii ale problemei clasice de colorare. Una dintre aceste extensii importante este multicolorarea grafurilor, care permite atribuirea mai multor culori unui singur vârf, reflectând astfel mai fidel realitatea problemelor din lumea reală unde o entitate poate necesita sau poate beneficia de multiple resurse simultan.