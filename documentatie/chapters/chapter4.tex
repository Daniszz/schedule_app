\chapter{Experimente și rezultate}

\section{Măsurători și metrici}
Inițial vom discuta despre cum arată instanțele de test in ceea ce privește parametri de bază ai algoritmilor: graful $G=(V,E)$, vectorul de culori $p$, numărul total de unități de timp/culori $D$, limita numărului de unități de resurse partajate/culori $l$ și vectorul de câștig $g$. Pentru generarea grafului se folosesc doi parametri: numărul n de vârfuri și o densitate a grafului d (probabilitate de avea o muchie între două noduri). Am considerat $n \in \{30,150,300\}$ și $d \in \{0.2, 0.5, 0.8\}$. Pentru fiecare combinație (n,d) sunt generate câte 10 instanțe. Numărul de culori $p_i$ a unui vârf $i$ este un număr întreg pozitiv ales aleatoriu în intervalul $[1,10]$. Pentru a simula scenariile reale unde câștigul este în strânsă legătură cu timpul de procesare ($p_i$), cu cât e un timp de procesare mai mare cu atât și câștigul este mai mare. Prin urmare $g_i = \beta \cdot p_i$, unde $\beta$ este ales aleatoriu în intervalul $[1,20]$. Limita de resurse $l$ este setată la 25.

În privința lui $D$ (numărul total de culori) am dori să aibă o valoarea ce nu permite colorarea completă pentru a vedea comportamentul algoritmilor cât mai bine.
Pentru a determina o valoare potrivită a lui $D$, a fost creat un script care, pentru fiecare combinație $(n, d)$, generează 100 de instanțe folosind algoritmul Greedy~\ref{alg:greedy_multicoloring}. Pentru fiecare $D$, se calculează procentul mediu de colorare, precum și intervalul [min, max] al procentelor de colorare obținute. Valoarea aleasă pentru testare este cel mai mare $D$ care nu a permis atingerea unui grad de colorare de $100\%$ în niciuna dintre instanțe. 

\begin{table}[H]
\centering
\caption{Rezultatele generării valori $D$}
\begin{tabular}{ccccc}
\toprule
$n$ & $d$ & $D$ & Procent colorat mediu & Interval\\
\midrule
30  & 0.2 & 24       & 85.5\% & [70.0\% $-$ 96.7\%] \\
30  & 0.5 & 42       & 83.7\% & [70.0\% $-$ 96.7\%]\\
30  & 0.8 & 68       & 82.0\% & [60.0\% $-$ 96.7\%]\\
150 & 0.2 & 63       & 85.1\% & [79.3\% $-$ 91.3\%]\\
150 & 0.5 & 132      & 84.6\% & [74.7\% $-$ 91.3\%]\\
150 & 0.8 & 239      & 82.8\% & [72.0\% $-$ 91.3\%]\\
300 & 0.2 & 101      & 85.8\% & [81.7\% $-$ 90.0\%]\\
300 & 0.5 & 228      & 85.8\% & [78.3\% $-$ 90.0\%]\\
300 & 0.8 & 423      & 84.7\% & [79.0\% $-$ 90.0\%]\\
\bottomrule
\end{tabular}\label{tab:rezultate_booktabs}
\end{table}
Import de menționat că deși valorile prezentate în tabel sunt rezultatul a 100 de instanțe, generarea grafului dar și algoritmul Greedy sunt realizate prin mecanisme predominat aleatorii, deci aceste date pot varia. Aceste aspecte aleatorii datorează și variața valorilor din intervale.

Algoritmii supuși instanțelor de test implică, pe lângă parametrii de bază, și o serie de parametri suplimentari specifici fiecărui algoritm, după cum urmează:
\begin{enumerate}
  \item \textbf{Greedy}~\ref{alg:greedy_multicoloring}: utilizează exclusiv parametrii de bază ai instanței, ceea ce explică simplitatea implementării și eficiența în ceea ce privește timpul de execuție.

  \item \textbf{Descent}~\ref{alg:descent_multicoloring}: necesită definirea unui număr maxim de iterații și a unui număr de restarturi. Aceste valori au fost stabilite prin rulari consecutive pentru fiecare combinație $(n,d)$, pana cand s-a gasit pragul in care creșterea celor două variabile nu produce nicio îmbunătățire (cu o limită de timp de $n$ secunde).
  \item \textbf{Tabu Search}~\ref{alg:tabu_search}: similar algoritmului Descent, acest algoritm primește un număr maxim de iterații, stabilit conform aceleiași logici de limitare temporală. În plus, include următorii parametri:
  \begin{itemize}
    \item dimensiunea numărului de mutări interzise ($tab$): setată la 3 pentru $n = 30$ și la 10 pentru $n \in \{150, 300\}$;
    \item Parametrul de diversificare $/$ procentul de decoloare ($b$): 20\%;
    \item numărul maxim de iterații fără îmbunătățire ($I_{\max}$): 150;
    \item numărul de noduri selectate pentru $R^{Exact}$ ($q$): 20;
    \item procentajul de vârfuri utilizat în generarea vecinătății~\ref{alg:generator}: 25\%;
    \item Probabilitate aplicări $R^{SD}$ în cadrul celei de a doua strategii ($\gamma$): 0{,}2.
  \end{itemize}
  Valorile acestor parametri au fost determinate în urma unei etape de ajustare, bazată pe rulări preliminare ale algoritmului pe un set reprezentativ de instanțe.
\end{enumerate}


Instanțele de test vor fi încadrate in 3 tabele distincte, grupate în funcție de  numărul de vârfuri $n$ astfel încât să se evidențieze comportamentul algoritmilor în cadrul instanțelor de dimensiuni reduse,medii și mari. Pentru fiecare combinație $(n,d)$ și pentru fiecare algoritm analizat: Greedy, Descent, TS1(Tabu Search cu strategia 1), TS2(Tabu Search cu strategia 2) sunt acordate două coloane: o coloană cu cel mai bun rezultat și o coloană cu rezultatul mediu din cele 10 instanțe. Rezultatele o să fie de forma unui triplet (scor obiectiv f1,scor obiectiv f2,scor obiectiv f3) însoțit de raportul dintre numărul de vârfuri colorate și numărul total de vârfuri.


\begin{table}[H]
\centering
\caption{Rezultate pentru $n = 30$}
\begin{tabular}{ccll}
\toprule
$d$ & Algoritm & Cel mai bun rezultat  & Media rezultatelor \\
\midrule
\multirow{4}{*}{0.2}
  & Greedy  & (1747, 5, 185), 22/30 & (1623.10, 2.70, 148.40), 20.8/30 \\
  & Descent & (1808,30,313), 25/30   & (1762.10, 30.90, 296.90), 24/30 \\
  & TS1     & (1808,22,243), 25/30    & (1771.40, 18.80, 242.00), 23.6/30 \\
  & TS2     & (1808,58,332), 25/30    & (1782.70,40.40,297.00), 23.6/30 \\
\midrule
\multirow{4}{*}{0.5}
  & Greedy  & (1614,7,170), 26/30     & (1555.30, 5.60, 169.30), 24.89/30 \\
  & Descent & (1704,40,464), 29/30    & (1671.50, 26.60, 365.70), 28/30 \\
  & TS1     & (1704,31,384), 29/30    & (1684.60, 30.30, 378.00),
  27.7/30 \\
  & TS2     & (1704,32,409), 29/30    & (1669.70,53.30,492.70), 28.7/30 \\
\midrule
\multirow{4}{*}{0.8}
  & Greedy  & (1617,7,169), 27/30     & (1588.70, 6.10, 198.80), 26.4/30 \\
  & Descent & (1644,43,749), 28/30    & (1631.40, 42.00, 699.20), 28/30 \\
  & TS1     & (1688,35,620) 29/30     &  (1646.30, 26.20, 473.10), 28/30 \\
  & TS2     &  (1644,38,556) 28/30    &  (1644.00, 43.80, 608.40), 28.1/30 \\
\bottomrule
\end{tabular}\label{tab:rezultate_n30}
\end{table}

\begin{table}[H]
\centering
\caption{Rezultate pentru $n = 30$}
\begin{tabular}{ccll}
\toprule
$d$ & Algoritm & Cel mai bun rezultat  & Media rezultatelor \\
\midrule
\multirow{4}{*}{0.2}
  & Greedy  & (8075, 17, 736), 138/150 & (7994.20, 25.40, 882.40), 134.7/150 \\
  & Descent & (8358,193,3259), 148/150   & (8335.20, 240.10, 3330.10), 148.5/150 \\
  & TS1     & (8364,77,1421), 149/150    & (8326.80, 64.40, 1298.60), 147/150 \\
  & TS2     & (8364,68,1537), 149/150    & (8341.70 ,68.50,1393.00), 148/150 \\
\midrule
\multirow{4}{*}{0.5}
  & Greedy  & (8517,53,1903), 129/150     & (8253.50, 43.70, 1608.70), 124.6/150 \\
  & Descent & (8862,296,7593), 147/150    & (8807.70, 284.40, 7148.60), 145.9/150 \\
  & TS1     & (8862,130,3335), 147/150    & (8802.50, 133.00, 3350.20), 144.9/150
  27.7/30 \\
  & TS2     & (8891,172,4249), 148/150    & (8829.20,155.50,3720.80), 145.4/150 \\
\midrule
\multirow{4}{*}{0.8}
  & Greedy  & (7655,42,2247), 131/150     & (7579.80, 41.80, 1904.30), 126.8/150 \\
  & Descent & (7992,280,11956), 149/150    & (7964.10, 250.90, 11422.90), 147.8/150 \\
  & TS1     & (7985,150,5995) 148/150     &  (7953.90, 144.50, 5258.40), 146.7/150 \\
  & TS2     &  (8006,212,7657) 150/150    &  (7991.30, 188.10, 6812.80), 148.5/30 \\
\bottomrule
\end{tabular}\label{tab:rezultate_n150}
\end{table}



\section{Interpretarea rezultatelor și concluzii parțiale}
\noindent
În cadrul experimentelor, timpul de execuție nu a fost utilizat ca metrică de performanță, ci doar ca reper pentru setarea parametrilor algoritmilor metaheuristici. Astfel, pentru algoritmii \textit{Descent} și \textit{Tabu Search}, numărul maxim de iterații (și, unde este cazul, numărul de restarturi) a fost calibrat astfel încât timpul total de execuție să nu depășească pragul de $10 \cdot n$ secunde, unde $n$ este numărul de vârfuri al grafului procesat. Prin urmare, timpul a fost uniformizat între instanțele de aceeași dimensiune și nu a reprezentat un criteriu de comparație între algoritmi. Accentul a fost pus pe calitatea soluțiilor obținute în acest interval prestabilit de timp.
