\chapter*{Contribuție}
\addcontentsline{toc}{chapter}{Contribuție}


Lucrarea de față pornește de la teoria multicolorării și se axează pe rezolvarea problemei de planificare a joburilor. În capitolele ce urmează, voi combina conceptele teoretice de colorare în scopul implementării unor algoritmi de optimizare, algoritmi ce vor fi utilizați într-o aplicație web care permite chiar și utilizatorilor non-tehnici să programeze joburile dorite.

Contribuțiile principale ale lucrării sunt următoarele:
\begin{itemize}
\item {\bf Studiul și adaptarea algoritmilor de multicoloring.} Am analizat problema matematică a multicolorării în contextul planificării joburilor și, pe baza acesteia, am dezvoltat și implementat trei algoritmi în Python. Algoritmii au fost proiectați astfel încât să optimizeze cerințele specifice problemei, fiind integrați într-o bibliotecă modulară, ușor de reutilizat, testat, extins și adaptat la noi cerințe. Perfor- manța acestora a fost evaluată printr-o serie de teste comparative, evidențiind avantajele și limitările fiecărui algoritm.

\item {\bf Dezvoltarea unei aplicații web.} Având în vedere și utilizatorii non-tehnici, am realizat o aplicație web cu o interfață grafică intuitivă și ușor de utilizat. Aceasta permite testarea și aplicarea algoritmilor de multicolorare în mod interactiv, în funcție de constrângerile introduse de utlizator.
\end{itemize}