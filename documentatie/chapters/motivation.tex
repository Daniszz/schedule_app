\chapter*{Motivație} 
\addcontentsline{toc}{chapter}{Motivație}

 Luând în considerare cerințele și scenariile tot mai complexe din lumea reală, se poate concluziona că planificarea tradițională, bazată pe colorarea clasică a grafurilor, nu mai este suficientă. Planificarea joburilor este considerată una dintre problemele centrale în domeniul sistemelor de calcul, al managementului de proiecte și al optimizării proceselor industriale. În toate aceste domenii, dificultatea principală constă în faptul că joburile nu mai pot fi tratate ca entități simple, care necesită o singură resursă la un anumit moment. Un job modern poate necesita simultan mai multe tipuri de resurse (procesor, memorie, rețea, stocare)  și poate beneficia de paralelizare pe mai multe unități de procesare. 


De ce este multicolorarea grafurilor un model mai potrivit? Răspunsul vine în mod natural: această metodă permite atribuirea mai multor „culori” (resurse) unui singur „vârf” (job). Astfel, se obține o modelare mai fidelă a cerințelor multiple ale fiecărui job și o gestionare mai flexibilă a constrângerilor de prioritate și dependență. 